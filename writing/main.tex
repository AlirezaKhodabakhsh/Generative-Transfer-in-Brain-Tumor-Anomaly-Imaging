\documentclass[12pt,onecolumn,a4paper]{article}
\usepackage{epsfig,graphicx,subfigure,amsthm,amsmath}
\usepackage{color,xcolor}     
\usepackage{xepersian}
\settextfont[Scale=1.2]{BZAR.TTF}
\setlatintextfont[Scale=1]{Times New Roman}



\begin{document}
\title{\lr{Unsupervised Anomaly Detection and Localization Via GAN}} 
\author{\lr{Alireza Khodabakhsh}}
\date{\today}
\maketitle

\section{\lr{Problem Statement}}
در نمادگذاری مسئله ای که با آن روبرو هستیم, تصاویر نرمال را با
$\mathbf{X}$
و تصاویر غیرنرمال را با 
$\mathbf{Y}$
نمادگذاری می‌کنیم.\\
همچنین بخش غیرسالم (در حوزه پیکسل) از $\mathbf{Y}$ را با 
$\mathbf{M} = [m_{ij}] ; m_{ij} \in [0,1]$
نمادگذاری می‌کنیم.
\\
اکثریت روش‌های موجود با تمرکز روی صحت تشخیص
\footnote{\lr{Detection}}
تصاویر نرمال و غیرنرمال می‌پردازند
\cite{zhou2021proxy}
.
چنین روش‌هایی قابلیت مکان‌یابی
\footnote{\lr{Localization}}
 و یا بخش‌بندی
 \footnote{\lr{Segmentation}}
را ندارند و یا به صورت ساده با تفریق تصاویر
$\mathbf{X}$
و
$\mathbf{Y}$
به این قابلیت دست پیدا می‌کنند
\cite{schlegl2017unsupervised} 
که به علت سادگی اطلاعات دقیقی و درستی از 
$\mathbf{M}$
را در اختیار ما قرار نمی‌دهد.
\\
روش‌هایی از جمله
\cite{yeh2017semantic}
نیز وجود دارد که با استفاده از شبکه‌های مولد متخاصم
\footnote{\lr{GAN}}
و همچنین در اختیار داشتن
$\mathbf{M}$
به بازسازی بخش ناسالم از
$\mathbf{Y}$
می پردازد که علارغم موفقیت‌هایی که در این بازسازی دارد, بایستی
$\mathbf{M}$
حتما در اختیار الگوریتم قرار گیرد که چنین محدودیتی در دادگان تصاویر پزشکی هزینه‌بر و گاها انجام نشدنی است. همچنین استفاده از گن نیازمند تعداد دادگان زیادی بوده که چنین امری در مسائل پزشکی چالش برانگیز است.
\\
در این مقاله با استفاده از
\lr{GAN}
و 
\lr{Transfer Learning}
سعی بر تشخیص مسائل \lr{AD} را داریم تا بر چالش محدودیت تعداد تصاویر آموزش غلبه کنیم.
همچنین در مکان‌یابی/بخش‌بندی (تخمین
$\mathbf{M}$
)
با استفاده از یک رویکرد جدید و نو به یادگیری
\footnote{\lr{Learning}}
 این بخش بپردازیم تا برخلاف مقالات \cite{schlegl2017unsupervised} بتوانیم دقت و اطالاعات سطح بالایی در استخراج شود.
 \\
 در نهایت به صورت خلاصه, در با در اختیار داشتن
 $\mathbf{X}$
 شبکه گن ای را آموزش خواهیم داد که در فاز تست علاوه‌بر تشخیص تصاویر سالم و ناسالم, قسمت ناسالم از 
$\mathbf{Y}$
  که با 
$\mathbf{M}$

 نمادگذاری شده است را بازسازی و مکایابی/قسمت‌بندی کند.
 


\begin{latin}
\bibliographystyle{plain} % We choose the "plain" reference style
\bibliography{refs}
\end{latin}



\end{document}